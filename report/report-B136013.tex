\documentclass[12pt,a4paper]{article}

\usepackage{epcc}
\usepackage{graphicx}
\usepackage{listings}
\usepackage{color}
\usepackage{amsmath}

\definecolor{mygreen}{rgb}{0,0.6,0}
\definecolor{mygray}{rgb}{0.5,0.5,0.5}
\definecolor{mymauve}{rgb}{0.58,0,0.82}

\lstset{ 
	backgroundcolor=\color{white},   % choose the background color; you must add \usepackage{color} or \usepackage{xcolor}; should come as last argument
	basicstyle=\footnotesize,        % the size of the fonts that are used for the code
	breakatwhitespace=false,         % sets if automatic breaks should only happen at whitespace
	breaklines=true,                 % sets automatic line breaking
	captionpos=b,                    % sets the caption-position to bottom
	commentstyle=\color{mygreen},    % comment style
	deletekeywords={...},            % if you want to delete keywords from the given language
	escapeinside={\%*}{*)},          % if you want to add LaTeX within your code
	extendedchars=true,              % lets you use non-ASCII characters; for 8-bits encodings only, does not work with UTF-8
	frame=single,	                   % adds a frame around the code
	keepspaces=true,                 % keeps spaces in text, useful for keeping indentation of code (possibly needs columns=flexible)
	keywordstyle=\color{blue},       % keyword style
	language=C,                 	 % the language of the code
	morekeywords={*,...},            % if you want to add more keywords to the set
	numbers=left,                    % where to put the line-numbers; possible values are (none, left, right)
	numbersep=5pt,                   % how far the line-numbers are from the code
	numberstyle=\tiny\color{mygray}, % the style that is used for the line-numbers
	rulecolor=\color{black},         % if not set, the frame-color may be changed on line-breaks within not-black text (e.g. comments (green here))
	showspaces=false,                % show spaces everywhere adding particular underscores; it overrides 'showstringspaces'
	showstringspaces=false,          % underline spaces within strings only
	showtabs=false,                  % show tabs within strings adding particular underscores
	stepnumber=5,                    % the step between two line-numbers. If it's 1, each line will be numbered
	stringstyle=\color{mymauve},     % string literal style
	tabsize=2,	                     % sets default tabsize to 2 spaces
	title=\lstname                   % show the filename of files included with \lstinputlisting; also try caption instead of title
}

\usepackage{hyperref}
\hypersetup{
	colorlinks=true, %set true if you want colored links
	linkcolor=black,  %choose some color if you want links to stand out
}

\newcommand{\sectionVspacing}{\vspace{15pt}} 


\begin{document}

\title{Threaded Programming Coursework Part 1}
\author{Exam number B136013}
\date{\today}

\makeEPCCtitle

\thispagestyle{empty}

\newpage

\tableofcontents

\newpage



\section{Introduction}
This experiment measures the running time and speedup of a C program for multiple configurations using openMP on Cirrus supercomputer. The source code contains two (2) for loops, initialization procedures and time reporting commands. We used a variety of schedule methods on the loops which are implemented in openMP API multithread programming library. Next, we pick the fastest scheduling options for each loop and run them using different number of threads, in order to observe which configuration had the best speedup results.

\sectionVspacing

\section{Visual Representation}
	
\subsection{Execution Time}
In the following section, there are graphs for the execution time and speed up of the given loops using different scheduling configurations like static, dynamic, guided against chunksizes such as 1, 2, 4, 8, 16, 32, 64.

The active number of threads has been set to four (4) for all of the following executions in order to choose the best schedule taking into account only the running time.

In Figure 1 and 2 are the execution time results in seconds of loop 1 and 2 for each one of the different scheduling configurations. As a result, in this series of executions the best scheduling options are shown in Table 1:

\begin{table}[h]
	\begin{center}
		\begin{tabular}{||l|c|l||}
			\hline
			{\bf Loop No} & {\bf Schedule}\\
			\hline
			Loop 1         &  schedule(guided, 4)\\
			Loop 2         &  schedule(dynamic, 16)\\
			\hline
		\end{tabular}
	\end{center}
	\caption{Best scheduler options on four (4) threads}
	\label{simple_table}
\end{table}

\begin{figure}[ht]
	\centering
	\includegraphics[scale=0.6]{../screenshots/threads4_loop1.eps}
	\caption{Running Time Loop 1 with 4 Threads}
	\label{loop1-threads4}
\end{figure}

\begin{figure}[ht]
	\centering
	\includegraphics[scale=0.6]{../screenshots/threads4_loop2.eps}
	\caption{Running Time Loop 2 with 4 Threads}
	\label{loop2-threads4}
\end{figure}

\clearpage

\subsection{Speedup}
Speedup occurs dividing T1/Tp where Tp is the execution time of the process running over p threads, and T1 is the same for one (1) thread.

As regards the speepup measurements, we picked up the best scheduling options for each loop as defined in the first two (2) lines of the source code shown in Listing 1: loops.c based on Table 1 results. Now, we are going to measure the speedup of the experiment running against different numbers of threads including 1, 2, 4, 6, 8, 12 and 16.

\lstinputlisting[language=C, firstline=2, lastline=3, caption=loops.c]{../template/loops_parallel-B136013.c}

Figure 3 and 4 depict the results of speedup in the application. As expected, when we increase the number of threads running the experiment, speedup increases or stays constant over a point.

\begin{figure}[h]
	\centering
	\includegraphics[scale=0.6]{../screenshots/speedup_loop1.eps}
	\caption{Speedup Loop 1 with schedule guided and chunksize 4}
	\label{Speedup-Loop1}
\end{figure}

\begin{figure}[ht]
	\centering
	\includegraphics[scale=0.6]{../screenshots/speedup_loop2.eps}
	\caption{Speedup Loop 2 with schedule dynamic and chunksize 16}
	\label{Speedup-Loop2}
\end{figure}

\clearpage

\section{Explanation}

\subsection{Loop 1}

Loop 1 updates all of the values upper of the main diagonal (without the main diagonal) in a two-dimensional table. The given implementation assess more calculations to the outer loop's first iterations. In general, because of the small size of the problem monitor time is under half of a second, so the noise from the system makes a big difference in the timing.

\subsubsection{Scheduling option}
In terms of Loop 1 the best scheduler option running on 4 threads is (guided, 4) following by (guided, 8) and (guided, 1) with a slight running time difference of 0,001 and 0,004 seconds accordingly.

Dynamic and guided schedules perform better due to the fact that in this case, they take approximatelly the same load in total. For example, if thread 1 takes chunk 1 which contains a lot of calculations, that means whilst it is working on chunk 1 the other threads share the remaining load, not waiting for thread 1 to handle more chunks.

In contrast, static scheduling performs poorly because of the unbalanced loops. Assesing the "heavy" chunks to the first threads and the "light" to the last ones, makes for example, thread 0 and 1 to do more iterations than thread 2 and 3.

\subsubsection{Threads configuration}
We observe that Loop 1 performed better with 16 threads. To be more specific, as long as we use more threads the performarce of Loop 1 increases in a linear fashion. This makes sense because there are no data dependencies between different iterations of the loop. Also, as we increase the running threads on guided scheduling the chunksize becomes smaller. This means that the heavy workload (first outer loop iterations) will be divided fairly well to the active threads.

\subsection{Loop 2}

As regards Loop2 we can observe with a number of tests that it works heavily for the first iterations, due to the fact that for i's near to zero jmax[i] equals to N most of the times. In that case, nested loops perform more iterations. On the contrary, when i is big enough then jmax[i] becomes one (1) for the majority of the outer iterations, therefore the two nested loops are executed just once each.

\subsubsection{Scheduling option}
In terms of Loop 2, the fastest configuration was (dynamic, 16) followed by (dynamic, 8) and (static, 8). This is the reason why threads share the work load of both the heavy (small i's) and light (big i's) iterations.

On the one hand, static and dynamic are the fastest configurations due to the fact that they take the same chunksize to calculate. The only difference is that dynamic is a little bit faster because the chunks are not statically allocated to each threads. In case that, whenever a thread finishes its task it can allocate a new one from the available chunks.

On the other hand, guided scheduling takes the same time to be executed independent of the chunksize. This happens because the threads that will execute the first iterations will take the same work load in each case. Taking into consideration that the first iterations are computationally intense, that sums up to high and constant latency for this configuration.

\subsubsection{Threads configuration}
Figure 4 depicts that Loop 2 performed better with more than 6 threads, but also it had the same results for configurations such as 6, 8, 12 and 16 threads. This happened possibly because more than 6 threads are not required to improve performance in the specific problem. In this situation with schedule(dynamic, 16), 6 threads are enough to handle the "heavy" iterations. Adding more threads than that it will have no speedup impact because the work load is obsolete, so further distribution is not needed.

\sectionVspacing

\section{Conclusion}
In conclusion the decision about scheduling selection and number of threads that need to be deployed is not an obvious choice. It always depends on the problem and how the work load is distributed between iterations. Analysis and measurements using different configuration options will guide the developer on what is the best approach of solving a problem effiently.

\end{document}
