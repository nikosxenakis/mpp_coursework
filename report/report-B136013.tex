\documentclass[12pt,a4paper]{article}

\usepackage{epcc}
\usepackage{graphicx}
\usepackage{listings}
\usepackage{color}
\usepackage{amsmath}

\definecolor{mygreen}{rgb}{0,0.6,0}
\definecolor{mygray}{rgb}{0.5,0.5,0.5}
\definecolor{mymauve}{rgb}{0.58,0,0.82}

\lstset{ 
	backgroundcolor=\color{white},   % choose the background color; you must add \usepackage{color} or \usepackage{xcolor}; should come as last argument
	basicstyle=\footnotesize,        % the size of the fonts that are used for the code
	breakatwhitespace=false,         % sets if automatic breaks should only happen at whitespace
	breaklines=true,                 % sets automatic line breaking
	captionpos=b,                    % sets the caption-position to bottom
	commentstyle=\color{mygreen},    % comment style
	deletekeywords={...},            % if you want to delete keywords from the given language
	escapeinside={\%*}{*)},          % if you want to add LaTeX within your code
	extendedchars=true,              % lets you use non-ASCII characters; for 8-bits encodings only, does not work with UTF-8
	frame=single,	                   % adds a frame around the code
	keepspaces=true,                 % keeps spaces in text, useful for keeping indentation of code (possibly needs columns=flexible)
	keywordstyle=\color{blue},       % keyword style
	language=C,                 	 % the language of the code
	morekeywords={*,...},            % if you want to add more keywords to the set
	numbers=left,                    % where to put the line-numbers; possible values are (none, left, right)
	numbersep=5pt,                   % how far the line-numbers are from the code
	numberstyle=\tiny\color{mygray}, % the style that is used for the line-numbers
	rulecolor=\color{black},         % if not set, the frame-color may be changed on line-breaks within not-black text (e.g. comments (green here))
	showspaces=false,                % show spaces everywhere adding particular underscores; it overrides 'showstringspaces'
	showstringspaces=false,          % underline spaces within strings only
	showtabs=false,                  % show tabs within strings adding particular underscores
	stepnumber=5,                    % the step between two line-numbers. If it's 1, each line will be numbered
	stringstyle=\color{mymauve},     % string literal style
	tabsize=2,	                     % sets default tabsize to 2 spaces
	title=\lstname                   % show the filename of files included with \lstinputlisting; also try caption instead of title
}

\usepackage{hyperref}
\hypersetup{
	colorlinks=true, %set true if you want colored links
	linkcolor=black,  %choose some color if you want links to stand out
}

\newcommand{\sectionVspacing}{\vspace{15pt}} 


\begin{document}

\title{Message Passing Programming Coursework Assignment}
\author{Exam number B136013}
\date{\today}

\makeEPCCtitle

\thispagestyle{empty}

\newpage
\clearpage

\tableofcontents

\newpage
\clearpage

\section{Introduction}
What is the project? 2d domain decomposition

\sectionVspacing

\section{Project Specification}
\subsection{Description}
Boundary conditions: horizontal->sawtooth vertical->periodic
Terminate condition: D parameter

\sectionVspacing

\section{Analysis}

\subsection{Tools}
Language used: C
\subsection{MPP API}
Tools from MPP (derived types, Async communication, virtual topologies)
\subsection{Design}
brief description of the design and implementation of your MPI program
\subsection{MPI Scatter}
\subsection{MPI Gather}

\sectionVspacing

\section{Evaluation}
Platform: backend of cirrus (.pbs), compiled using -O3
Build, Run and Submit job: in Readme.md

\subsection{Correctness}
Test: parallel code produces the same output as the serial

how long the code needs to run in order to give a reasonable assessment of its performance and/or correctness
It is not necessary to run all and be correct

\subsection{Performance}

Timing: exclude I/O from timing The timing starts before the scatter and ends after the gather
Graphs: demonstrate that the performance of the code improves as you increase the number of processes.
	 processes
	 Input variety: performance change input(problem) size and see how it scales
	 Speedup of strong scaling: performance metrics speed up vs ideal (linear) speedup 
	 average time per iteration

\sectionVspacing

\input{conclusion.tex}

\end{document}

% \begin{table}[h]
% 	\begin{center}
% 		\begin{tabular}{||l|c|l||}
% 			\hline
% 			{\bf Loop No} & {\bf Schedule}\\
% 			\hline
% 			Loop 1         &  schedule(guided, 4)\\
% 			Loop 2         &  schedule(dynamic, 16)\\
% 			\hline
% 		\end{tabular}
% 	\end{center}
% 	\caption{Best scheduler options on four (4) threads}
% 	\label{simple_table}
% \end{table}

% \begin{figure}[ht]
% 	\centering
% 	\includegraphics[scale=0.6]{../screenshots/threads4_loop2.eps}
% 	\caption{Running Time Loop 2 with 4 Threads}
% 	\label{loop2-threads4}
% \end{figure}
%
% \lstinputlisting[language=C, firstline=2, lastline=3, caption=loops.c]{../template/loops_parallel-B136013.c}
